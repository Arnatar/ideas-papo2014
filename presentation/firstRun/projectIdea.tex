\section{Projekt-Idee}
\subsection{Definition}
\begin{frame}{Idee, wie? Was wollen wir?}
	\begin{itemize}
		\item<1> Gedanke (Ich habe Hunger)
		\item<2> Konzept (Glühbirne)
		\item<3> Weltanschauung (naturwissenschaftliche Erklärung von Sachverhalten)
	\end{itemize}
\end{frame}

\subsection{Ziel}
\begin{frame}{Leitgedanke}
	\begin{block}{}
		\large{Simulation der Verbreitung von Ideen innerhalb einer geschlossenen Gruppe von n Rezipienten}
	\end{block}
\end{frame}

\subsection{Eigenschaften}
\begin{frame}{Was ist eine Idee?}
	\begin{block}{Eigenschaften einer Idee}
		\begin{itemize}[<+->]
			\item Qualität
			\item Komplexität
			\item "Weltanschauungswert"
		\end{itemize}
	\end{block}	
\end{frame}

\begin{frame}{Annahmen}
	\uncover<1>{	
	\begin{block}{}
		Qualitätswert definiert einen Wertebereich für die Komplexität
	\end{block}}
	\uncover<2>{
	\begin{block}{}
		Es existiert kein sofortiges "0 auf 100" der Komplexität
	\end{block}}
\end{frame}

\begin{frame}{Vermittelbarkeit}
	\begin{block}{}
		Vermittelbarkeit ergibt sich aus den Zusammenhängen der Werte.
	\end{block}
	\uncover<2-4>{
	\textbf{Beispiele hierfür:}
	\begin{itemize}
		\item<2-4> Vermittelbarkeit nimmt mit zunehmender Komplexität ab
		\item<3-4> Je höher die Qualität, desto vermittelbarer
		\item<4> Je ähnlicher die Weltanschauungswerte von Mensch und Idee sind, desto vermittelbarer
	\end{itemize}}
\end{frame}
