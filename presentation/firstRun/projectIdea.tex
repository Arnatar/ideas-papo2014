\section{Projekt-Idee}
\subsection{Definition}
\begin{frame}{Idee, wie? Was wollen wir?}
	\begin{itemize}
		\item<1> Gedanke (Ich habe Hunger)
		\item<2> Konzept (Glühbirne)
		\item<3> Weltanschauung (naturwissenschaftliche Erklärung von Sachverhalten)
	\end{itemize}
\end{frame}

\subsection{Eigenschaften}
\begin{frame}{Was ist eine Idee?}
	\begin{block}{Eigenschaften einer Idee}
		\begin{itemize}[<+->]
			\item Überzeugungskraft
			\item Vermittelbarkeit
		\end{itemize}
	\end{block}	
\end{frame}

\begin{frame}{Zusammenhänge}
	\begin{itemize}[<+->]
		\item Es existiert kein sofortiges 0 auf 100
		\item Überzeugungskraft ist umgebungsgebunden
		\item (Von ''oben") Je überzeugender, desto komplexer
		\item Vermittelbarkeit nimmt mit zunehmender Komplexität ab
	\end{itemize}
\end{frame}

\subsection{Ziel}
\begin{frame}{Simulation}
	\begin{block}{}
		Verbreitung von Ideen innerhalb einer geschlossenen Gruppe von n Rezipienten
	\end{block}
\end{frame}